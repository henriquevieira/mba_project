%% Resumo.tex
% ---
% Resumo
% ---
\setlength{\absparsep}{18pt} % ajusta o espaçamento dos parágrafos do resumo		
\begin{resumo}
	\begin{flushleft} 
			\setlength{\absparsep}{0pt} % ajusta o espaçamento da referência	
			\SingleSpacing 
			\imprimirautorabr~ ~\textbf{\imprimirtitulo}.	\imprimirdata. \pageref{LastPage}p. 
			%Substitua p. por f. quando utilizar oneside em \documentclass
			%\pageref{LastPage}f.
			\imprimirtipotrabalho~-~\imprimirinstituicao, \imprimirlocal, \imprimirdata. 
 	\end{flushleft}
\OnehalfSpacing 			
Os documentos de patentes são largamente utilizados por empresas para a construção estratégias de produção, marketing e desenvolvimento. O tempo gasto em pesquisa é influenciado pela quantidade de patentes em que os pesquisadores deverão ler e estudar para gerar algum insight significativo. Neste trabalho nos propomos a desenvolver uma metodologia de classificação de documentos de patentes por áreas de interesse facilitando o encontro de patentes mais relevantes. Faremos uso de técnicas de processamento de linguagem natural e compararemos os modelos mais utilizados na classificação de textos.
 

 \textbf{Palavras-chave}: Classificação, Documento de patente, RandomForest, SVM, Naive Bayes, Dicionário, Processamento de Linguagem Natural
\end{resumo}