%% Abstract.tex
% ---
% Abstract
% ---
\autor{Silva, M. J.}
\begin{resumo}[Abstract]
 \begin{otherlanguage*}{english}
	\begin{flushleft} 
		\setlength{\absparsep}{0pt} % ajusta o espaçamento dos parágrafos do resumo		
 		\SingleSpacing 
 		\imprimirautorabr~ ~\textbf{\imprimirtitleabstract}.	\imprimirdata.  \pageref{LastPage}p. 
		%Substitua p. por f. quando utilizar oneside em \documentclass
		%\pageref{LastPage}f.
		\imprimirtipotrabalho~-~\imprimirinstituicao, \imprimirlocal, 	\imprimirdata. 
 	\end{flushleft}
	\OnehalfSpacing 
   Patent documents are widely used by companies to build production, marketing and development strategies. The time spent on research is influenced by the number of patents that researchers must read and study to generate some meaningful insight. In this work we propose to develop a methodology for classifying patent documents by areas of interest, facilitating the finding of more relevant patents. We will make use of natural language processing techniques and compare the most used models in the classification of texts.

   \vspace{\onelineskip}
 
   \noindent 
   \textbf{Keywords}: Classification, Patent Document, RandomForest, SVM, Naive Bayes, Dictionary, Natural Language Processing
 \end{otherlanguage*}
\end{resumo}
