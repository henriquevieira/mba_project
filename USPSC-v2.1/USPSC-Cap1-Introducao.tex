%% USPSC-Introducao.tex

% ----------------------------------------------------------
% Introdução (exemplo de capítulo sem numeração, mas presente no Sumário)
% ----------------------------------------------------------
\chapter[Introdução]{Introdução}


\section{Apresentação}	

No desenvolvimento geral de produtos, a pesquisa por documentos de patentes visa garantir o não infringimento de propriedades intelectuais que ainda não estão em domínio público \cite{Breitzman2002}.  O sistema de patentes é um conjunto de medidas utilizados para visar o retorno do valor privado investido ao valor social de suas invenções, fornece aos inventores um período temporário de poder de mercado, recuperando os custos de seus investimentos na pesquisa \cite{Williams2017}. De acordo com o \textit{World Intellectual Property Indicator 2017}, em 2016, o número de documentos de patente excedeu 3 milhões pela primeira vez, um aumento de 8.3\% \cite{Li2018}. Em uma pesquisa de documentos de patente, documentos relacionados a tecnologia, economia e jurídico são tratadas, classificadas e analisadas para se obter uma grande vantagem técnica e comercial \cite{Li2018}.

A classificação de documentos é o processo de classificação de um documento em uma ou mais categorias predefinidas, desempenhando um papel importante no gerenciamento e busca de temas \cite{Anne2017}. A automatização da classificação de documentos a partir do uso de aprendizado de máquina, pode rotular documentos a um tema único, mas a rotulagem em dois ou mais temas ainda é relativamente um problema desafiador \cite{Anne2017}.

De acordo com Shahid et al (2020), a classificação de documentos de patente em temas e a atribuição de valor de relevância para estes temas, permitem ao pesquisador filtrar as patentes que o interessa e reduzindo o escopo de análise. Nesse trabalho, realizou-se a construção de uma matriz de valores de \textit{term frequency - inverse document frequency} (TF-IDF), notações e peso ponderado por BestMatch25 (BM25), que posteriormente foi testado em diferentes classificadores, classificando os documentos de patente em cada assunto.
Vide Anne et al (2017), identificou uma matriz de métodos a serem aplicados com os modelos k-Nearest Neighbors (kNN),  Support Vector Machine (SVM), Random Forest e J48. Os principais passos para essa pesquisa foram técnicas de seleção de características, com uso de ganho de informação e correlação para efetividade do classificadores.

Destes dois estudos, foi observado que a adição de mais características para os modelos de classificação utilizados, a acurácia foi melhorada \cite{shahid2020}. E que obstáculos, como o desbalanceamento dos dados foram atenuados pela adição de novas características \cite{Anne2017}. Balancear a relação entre esses dois pontos é um desafio quanto a classificação de documentos de patente.

\section{Justificativa}
Como tratar, classificar e analisar documentos de patente havendo algumas centenas de documentos sobre um assunto específico? O método tradicional necessita de tempo e equipe para realizá-lo, apresentando um resultado com deficiências devido ao alto volume de documentos de patente a serem analisadas \cite{Li2018}. Hoje, já há portais web que oferecem ferramentas das quais algumas auxiliam ao pesquisador a reduzir essa pesquisa \cite{Abbas2014}, mas classificam os documentos em uma relevância geral. Esse resultado somente demonstra que dentro daquela amostra de documentos, uma visão macro sobre o tema que muitas vezes o pesquisador está em busca de um subtema, como quais mercados essa tecnologia está presente, quais os processos de produção desta tecnologia ou qual a formulação desse composto. 

\section{Problema}
Com o rápido crescimento de documentos de patente, torna-se urgente a questão de automatização da classificação de documentos de patente de forma acurada e rápida \cite{Zhu2020}. Os documentos de patente contêm um potencial conhecimento tecnológico na resolução de problemas no processo de fabricação, nos quais são de grande valor científico e tecnológico, no entanto, esse conhecimento está implícito em longos textos \cite{Li2018, Wang2016}. A classificação de documentos de patente em temas e subtemas utilizando de modelos de aprendizado  de máquina se beneficiaria do uso da extração de características uteis vindas do próprio documento \cite{Anne2017}. Observa-se que mais de 90\% das informações de científicas e tecnológicas estão em documentos de patente, e sua análise resultaria em decisões de negócio de sucesso \cite{Li2018}.

\section{Objetivo geral}
Este projeto se propõe a classificar documentos de patente por tema específico e subtemas de interesse do pesquisador, reduzindo o escopo de documentos de patentes a serem estudados à somente os mais relevantes para o que se procura.

\section{Objetivos específicos}
A classificação de documentos de patente envolverá o uso de técnicas de processamento de linguagem natural para o tratamento e preparação dos dados que serão usados no modelo de classificação por relevância que será desenvolvido. Este modelo usará inicialmente a medida estatística TF-IDF e avaliaremos outras medidas. Haverá a necessidade de criação de dicionários que auxiliem na classificação dos documentos de patente. E então será treinado um algoritmo para classificar os documentos de acordo com o tema. 

\section{Metodologia}
Realizaremos a obtenção de um conjunto de documentos de patente aplicado a agricultura através da ferramenta \textit{Free Patents Online} - FPO (https://www.freepatentsonline.com/). Não foi encontrado artigos ou materiais que fizessem essa aplicação para patentes relacionadas ao setor agronômico, para gerenciamento de patentes, desenvolvimento de produtos e descoberta de mercados. 
Faremos o uso do modelo de classificação baseado em florestas aleatórias, a vantagem desse modelo, é a flexibilidade para o uso em regressão e classificação, além da sua facilidade de interpretação do resultado obtido.  
A construção de dicionários será a partir de técnicas de Processamento de Linguagem Natural, elencando as palavras mais relacionadas a área. A analise, construção de dicionários e modelagem do modelos de regressão e classificação será feita na linguagem de programação Python.
